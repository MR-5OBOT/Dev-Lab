\documentclass[11pt]{article}

\usepackage{amsmath}
\usepackage{graphicx}
\usepackage{fancyhdr}
\usepackage{geometry}

\geometry{a4paper, margin=1in}
\pagestyle{fancy}

\begin{document}

\title{The consistency keys: A Full risk management system for forex and futures}
\author{MR5OBOT \\ \textit{"a 21 years old ict student"}}
\date{\today}
\maketitle
\tableofcontents

\begin{abstract}
\noindent Risk management is essential in finance and insurance, focusing on minimizing and controlling the impact of adverse events through risk identification, assessment, and prioritization. At its core, the discipline relies on mathematical models and methods, including stochastic processes, probability theory, and statistics, to predict and mitigate financial risks. This paper highlights the key mathematical tools used in financial risk management, particularly in measuring and managing credit risk. It underscores the importance of a multidisciplinary approach, combining mathematical finance and quantitative disciplines, to effectively address the challenges in risk management.
\end{abstract}

\begin{quote}
{\scriptsize
\noindent 
\textbf{\underline{Disclaimer:}} The content presented in this paper is provided for informational and educational pur-
poses only and should not be considered as financial or investment advice. The author of this paper is
not a licensed financial professional, and the information contained herein should not be interpreted as
a recommendation, endorsement, or solicitation to engage in any financial or investment activity. The
views and opinions expressed in this paper are those of the author and do not represent the views of
any financial institution, regulatory body, or other organization. The author assumes no responsibility
for any financial losses, damages, or consequences resulting from the use of the information provided
in this paper.
}
\end{quote}
\newpage

\section{Introduction}

\newpage

\section{Risk Management Framework}

\newpage

\subsection{Mathematical Models for Risk Assessment}

\newpage
% Example of figure next to text
  \begin{center}
  \end{center}


\section{Conclusion}

\end{document}

