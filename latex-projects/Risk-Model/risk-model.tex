\documentclass[10pt]{article}
\usepackage[utf8]{inputenc}
\usepackage{graphicx}
\graphicspath{{~/Pictures/latex/}} % setting the graphics path
\usepackage{fancyhdr}
\usepackage{caption} % for captions
\usepackage[margin=4cm]{geometry}
\geometry{top=0.5in}
\usepackage{array}
\usepackage{longtable}
\usepackage{amssymb}
\usepackage{enumitem} % For customizing itemize lists
\usepackage{tabularx}
\usepackage{dashrule}
\usepackage[most]{tcolorbox}
\usepackage{amssymb}
\usepackage{xcolor}
\usepackage{colortbl} % for coloring columns
\usepackage{adjustbox}
\usepackage{multirow}


\title{The Consistency Key: A Full Risk Management Model\\ for Forex \& Futures}
\author{MR5OBOT }
\date{\today}


\begin{document}
\maketitle
% \pagestyle{fancy}
% \tableofcontents
\noindent\hdashrule[0.2ex]{\linewidth}{0.1pt}{1mm} 

\newtcolorbox{notebox}{
    colback=gray!8, % Light grey background color
    colframe=gray!10, % Border color
    arc=0mm, % No rounded corners
    boxrule=0.2mm, % Border thickness
    left=2mm, % Padding on the left side
    right=2mm, % Padding on the right side
    top=1mm,
    bottom=1.5mm,
}

% Define a new tcolorbox style for the "Notes" section
\newtcolorbox{notesbox}{
    colback=yellow!30, % Light yellow background color
    colframe=yellow!10, % Slightly darker yellow border color
    boxrule=0.5pt, % Border thickness
    arc=2pt, % Rounded corners
    left=5pt, % Left padding
    right=5pt, % Right padding
    top=3pt, % Top padding
    bottom=3pt % Bottom padding
}

\newtcolorbox{violetbox}{
    colback=violet!10, % Light yellow background color
    colframe=violet!10, % Slightly darker yellow border color
    boxrule=0.5pt, % Border thickness
    arc=2pt, % Rounded corners
    left=5pt, % Left padding
    right=5pt, % Right padding
    top=3pt, % Top padding
    bottom=3pt % Bottom padding
}
\subsubsection*{What Does Risk Management Really Mean?}
\vspace{0.1cm}
\begin{notebox}
\small It's The amount of Risk the Trader is wailling to assume. It should be limited to 1\% of total equity or less. To calculate the Risk per trade use the following formula...
\end{notebox}

\begin{itemize}
  \small{  \item \textbf{Account balance} : $ 10000\$ $}
  \small{  \item  \textbf{Percentage Risk} : $ 1\% $ }

\begin{center}
  \small{ \textbf{ \underline{The Formula}}}
  \[ ( 10000\$ * 1\% ) = 10000 * 0.01 = 100\$ \]
\end{center}
\end{itemize}

\begin{violetbox}
\small Here's an example of bad risk Management with  a 8 losses in a row - If you think this not going to happen to you... you're going to be very disappointed friend.
\end{violetbox}

\begin{figure}[h]
\begin{center}
   \includegraphics[scale=0.3]{10.png}
  % \captionof{figure}{10\% Risk}
\end{center}
\end{figure}

\small{\begin{tcolorbox}[colback=gray!20!white,colframe=red!70!black,title=Question]
    Do you think you could deal with this drawdown and fell confident and you'd recover from it ?
\end{tcolorbox}}
\newpage

\center{\subsubsection*{$\rightarrow$ Why Sound \underline{Money}\hspace{0.01cm} \underline{Management} Is Crucial ...}}
\vspace{0.3cm}

\renewcommand{\arraystretch}{1.5} % table padding vertical space 
\setlength{\tabcolsep}{10pt} % table padding horizontal space

\begin{center}
\begin{tabular}{|c|c|c|}
\cline{2-3}
  \multicolumn{1}{c|}{} & \textbf{Scenario A (1\% Risk)} & \textbf{Scenario B (5\% Risk)} \\
\hline
\textbf{Equity - Start} & \$10.000 & \$10.000 \\
\hline
Lossing Trade \#1 & \$9,900.00 & \$9,500 \\
\hline                     
Lossing Trade \#2 & \$9,801.00 & \$9,025 \\
\hline                     
Lossing Trade \#3 & \$9,702.00 & \$8,574 \\
\hline                     
Lossing Trade \#4 & \$9,604.00 & \$8,145 \\
\hline                     
Lossing Trade \#5 & \$9,507.00 & \$7,738 \\
\hline
  Cumulative Loss - \$ & \textcolor{red}{( \$493 )} & \textcolor{red}{( \$2,262 )} \\
\hline
Cumulative Loss - \% & -4.93\% & 22.62\% \\
\hline
\end{tabular}
\end{center}

\subsection{Position Sizing}

% \begin{itemize} 
%   \item Risk per Trade = 1\% Max        \vspace{0.2cm}
%   \item Max Daily Trades = 2 Trades     \vspace{0.2cm}
%   \item Max daily loss = 1\%            \vspace{0.2cm}
%   \item Max weekly Trades = 4 Trades    \vspace{0.2cm}
%   \item max weekly Loss = 3\%           \vspace{0.2cm}  
% \end{itemize}

\renewcommand{\arraystretch}{1.5} % table padding vertical space 
\setlength{\tabcolsep}{10pt} % table padding horizontal space

\subsection{What Seperates Professional Traders From You?}
\vspace{0.3cm}
\begin{center}
  \small{Combinations of | [ Win/Loss Over 10 Trades ]} \\
  \small{With 1:1 Risk To Reward}
\end{center}

\begin{center}
\begin{tabular}{|c|c|c|}
  \hline
  WINS & LOSSES & WIN RATE \\ 
  \hline 
  10000 & NYKZN & 08:40AM \\  
  \hline
\end{tabular}
\end{center}
\vspace{0.1cm}


\begin{notesbox}
  \noindent | With this rules we can make our self in the right way for been consistent in the market.
\end{notesbox}


\footnote{Twitter [ @MR-5OBOT ]}
\end{document}
